\documentclass{article}
\usepackage{graphicx}
\textwidth 6.0 true in
\textheight 9 true in
\topmargin -1.0 true in
\headsep 0.375 true in
\oddsidemargin 0.0 true in
\evensidemargin 0.5 true in

\title{An Introduction to glmmBUGS Package}
\begin{document}
\maketitle

\large
\section{Introduction}
 


\section{Key Functions and Examples}
The primary function defined in package \textit{glmmBUGS} is the model-fitting function of the same time.
This function is \textit{glmmBUGS}. 

Consider the non-spatial model. We took the bacteria data from library \textit{MASS} as an example: 
Here is how the data looks like : 
\begin{verbatim}
  y ap hilo week  ID     trt
1 y  p   hi    0 X01 placebo
2 y  p   hi    2 X01 placebo
3 y  p   hi    4 X01 placebo
4 y  p   hi   11 X01 placebo
5 y  a   hi    0 X02   drug+
6 y  a   hi    2 X02   drug+
\end{verbatim}
\begin{itemize}
\item y : presence or absence: a factor with levels n and y. 
\item ap : active/placebo: a factor with levels a and p. 
\item hilo : hi/low compliance: a factor with levels hi amd lo. 
\item week : numeric: week of test. 
\item ID : subject ID: a factor. 
item trt: a factor with levels placebo, drug and drug+, a re-coding of ap and hilo. 
\end{itemize} 

We could fit a glmmBUGS model as: 
\begin{verbatim}
bacrag <- glmmBUGS(y ~ trt + week , data=bacteria, effects = "ID", family= "bernoulli") 
\end{verbatim}

The response variable \textit{y} is a facter, but in order to fit it into the glmmBUGS model , it need to be a number. Therefore, we change it to numeric named yInt, it will be a column contains 1 and 0. 1 denotes ''y'' and 0 denotes ''n''. There are three 
levels of treatment, one of them is ''drug+'', we change it to ''drugplus'' for fitting the glmmBUGS model.

After the above changes, the data set looks like: 

<<thedata>>=
head(bacteria)
@


\begin{verbatim}
 y ap hilo week  ID      trt yInt
 y  p   hi    0 X01  placebo    1
 y  p   hi    2 X01  placebo    1
 y  p   hi    4 X01  placebo    1
 y  p   hi   11 X01  placebo    1
 y  a   hi    0 X02 drugplus    1
 y  a   hi    2 X02 drugplus    1
\end{verbatim}

We fit the glmmBUGS model using: 
\begin{verbatim}
bacrag <- glmmBUGS(yInt ~ trt + week , data=bacteria, effects = "ID", family= "bernoulli") 
\end{verbatim}


<<raggedArray>>=
names(bacrag$ragged)
@

We will get three set of values from glmmBUGS fuction as: ragged, startingValues and pql. 
\textit{ragged} is a matrix that contains the data which can be used in winBUGS. 
\textit{startingValues} provides the starting values of the mcmc simulation runs by winBUGS. 
\textit{pql} used the method of \textit{glmmPQL} function in the \textit{MASS}, it fits a generalized linear mixed model via PQL.

After running \textit{glmmBUGS} function, we will use \texit{bugs} function in \textit{R2WinBUGS} package
to call the winBUGS as: 
\begin{verbatim}
library(R2WinBUGS)
bacResult = bugs(bacrag$ragged, getInits, parameters.to.save = names(getInits()),
 model.file="model.bug", n.chain=3, 
n.iter=1000, n.burnin=100, n.thin=10, program="winbugs", debug=T)
\end{verbatim}


A winbugs window will open and starts simulating the posterior distribution after running the \textit{bugs} command in R. 

All the infomation are now stored in the bacResult, but it is hard to get out, we need to use  \textit{restoreParams} function to restore the parameters as: 

\begin{verbatim}
bacParams = restoreParams(bacResult, bacrag$ragged) 
\end{verbatim}

Then we used \textit{checkChain(bacParams)} to do the chain plot we get from the mcmc simulation as:
\begin{verbatim}
checkChain(bacParams)
\end{verbatim}

The plot is like: 

\begin{figure}[ht]
\begin{center}
\includegraphics[scale=0.3]{chain.pdf}
\caption{A chain plot of the posterial distribution from mcmc simulation.}\label{singleellipse}
\end{center}
\end{figure}

The three lines are converges, this means the simulation result are stable. 







Consider the following model, where $Y_{ijk}$ is the number of cancer cases from individual i from class j in school k.  

\[Y_{ijk} \sim  Poisson(\lambda_i)\]
\[\log(\lambda_i) = \mu + U_i + V_{ij}\]
\[U_i \sim GMRF(0, \sigma_U^2)\]
\[V_{i} \sim iid N(0, \sigma_V^2)\] 

Here there are covariates which apply to each of the three levels, and random effects at the school and class level. 
If \code{data} is a data frame with one line per individual, the following would impliment this model: 
\begin{verbatim}
forBugs = glmmBUGS(formula=Observed + logExpected ~ 1,
  effects="CSDUID",   family="poisson", spatial=popDataAdjMat,
  data=ontario)
\end{verbatim}
To aid in convergence, the bugs model is actually the following:

\deqn{\log(\mu_i) = age_{ijk} \beta + V_{ij}}{log(mu_i) = age_ijk beta +  V_ij }

\deqn{V_{ij} \sim N(U_i + classSize_{ij} \alpha , \nu^2)}{V_ij ~ N(U_i + classSize_ij alpha , nu^2)}

\deqn{U_i \sim N(\delta + schoolCategory_i \gamma, \sigma^2)}{U_i ~ N(intercept + schoolCategory_i gamma, sigma^2)}



\section{Summary}

\section{Acknowledgments}


\section{Bibliography}


\end{document}




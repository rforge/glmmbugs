\documentclass[12pt]{letter}

\usepackage[margin=1in]{geometry}

\address{
Lutong Zhou\\ Population  Studies and Surveillance\\
Cancer Care Ontario\\ 620 University Avenue\\Toronto, ON\\M5G 2L7\\
\\
{\tt carly\_zhou@hotmail.com}
}

\signature{Lutong Zhou}

\begin{document}

\begin{letter}{
Dr.\ Heather Turner\\
Heather.Turner@warwick.ac.uk
}

\opening{Dear Dr.\ Turner,} 



Please thank the referees for the valuable comments on our glmmBUGS paper. We've made changes according to the comments as follows:

\noindent{\bf Reviewer 1}

\begin{enumerate}
	\item No, the program can't handle lmer syntax, though the way random effects and the model formula are specified and handled should be improved and thank you for the suggestion. 
	\item overdispersed models could be specified by creating a column of unique observation identifiers and specifying this as a random effect.  
	\item Currently only random intercepts are allowed for.  Users would have to edit the bugs model file to allow for random slopes. 
\end{enumerate}
A paragraph on possible enhancements has been added, along with a call for contributors!

\noindent{\bf Reviewer 2}

\begin{enumerate}
\item I've included the two R packages (MCMCglmm and glmmGibbs) into the discussion section and put the information into the bibiliography	 section.  The advantage (and disadvantage!) of glmmBUGS with respect to the other packages is the use of WinBUGS for model fitting.

\item We are considering writing a function to automate the whole process. The argument against such a function is that users shouldn't be encouraged to accept the default priors and starting values. 

\item I've added the model file argument in the bugs and glmmBUGS functions.

\item The post-WinBUGS glmmBUGS functions are more clearly indicated.

\item  I've tidy up the R codes, so that it won't print outside the margins.
\end{enumerate}


\closing{Regards,}


\end{letter}

\end{document}
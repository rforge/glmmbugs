\documentclass[12pt]{article}

\usepackage[margin=1in]{geometry}

\begin{document}

\thispagestyle{empty}

\vspace*{10pt}

{\flushright

\begin{minipage}{3in}
Lutong Zhou\\ Population  Studies and Surveillance\\
Cancer Care Ontario\\ 620 University Avenue\\Toronto, ON\\M5G 2L7\\
\\
{\tt carly\_zhou@hotmail.com}
\\ \\
\today
\end{minipage}

}

\vspace{10pt}

\noindent Dear Dr.\ Turner, 

\vspace{10pt}

Thank you for your the valuable comments on our glmmBUGS paper. We've made changes according to the comments as follows:
Reviewer 1
\begin{enumerate}
\item I've included the two R packages (MCMCglmm and glmmGibbs) into the discussion section and put the information into the bibiliography	 section.  Working via WinBUGS is the difference between  glmmBUGS and the other packages, which is both an advantage and disadvantage.

\item We are considering generating a super function which included all the functions from the glmmBUGS package.  

\item I've added the argument "model.file = "model.bug" in the bugs function and a comment that it can be omitted.

\item I've added a section "pre and post-WinBUGS command" between the two examples, outline the usage of the functions briefly and when to use those functions.

\item  I've tidy up the R codes, so that it won't print outside the margins.
\end{itemize}


{\flushleft

\begin{minipage}{3in}
\vspace{10pt}
Regards,

\vspace{10pt}

Lutong Zhou

\end{minipage}

}


\end{document}